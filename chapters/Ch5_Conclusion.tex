%!TEX root = ../Thesis.tex
\chapter{Conclusions}
\label{sec:conclusions}

\clearpage

This PhD project has presented a framework for applications of minute-scale wind forecasts, and outlined various methods used to generate them in real time.

A collaborative workshop on the topic was convened through the IEA platform which brought together users and providers of wind power forecasts to exchange ideas and knowledge. The main outcomes have been reported in an open access journal article, which presents a broad overview of the field.

Applications of minute-scale wind forecasts mainly focus on three key areas: wind farm control, grid regulation and balancing, and trading in intrahour electricity markets.

In contrast to traditional closed-loop feedback control systems which adapt to immediate conditions at the turbine or wind farm, advanced control strategies using inputs from forecasts on the second and minute scale allow for preemptive optimization in anticipation of impending conditions. This acts to increase total production and decrease fatigue loads, leading to an expected extension of the turbine's lifetime and increased revenues for the owner. Short range nacelle lidars have become the standard for providing second-scale previews for individual wind turbine control, but lack the spatial coverage and measurement range needed for use in farm level control on the minute scale. Long range variants including high energy pulsed lidar or Doppler radar are however capable of measurement distances up to 5-30 km (design dependent) and are thus seen as suitable instruments for providing inputs to forecast models used for medium to long range control (i.e. in the range of 10-seconds to 10-minutes). The primary stakeholders for these applications are wind turbine manufacturers and plant operators.

The second major application lies within grid integration of wind power and is mainly applicable to national and regional transmission system operators (TSOs), balancing authorities (BA), and industrial participants in regulating roles (i.e. those supplying balancing capacity). The requirement of continuous physical balance between supply and demand in the electrical grid is strained by increased variability and production uncertainty introduced by the wind power generators. Imbalances between scheduled and realized production need to be accounted for by balancing controls which can occur either on the demand side (e.g. demand response or load shedding) or on the supply side (e.g. hydro or gas turbine governing, storage actions, wind power curtailment). Primary controls are normally automated actions on the seconds-scale and thus are not relevant in this context. However, secondary and tertiary balancing include both automated and manual actions with response times on the minute-scale. Improved wind power forecasts on the minute-scale can therefore reduce balancing costs and possibly enable wind power producers to participate in balancing roles in the future.

The third key application relates to the trading of power in wholesale electricity markets. Market operators such as NordPool and EPEX offer spot trading of electric power on day-ahead and intraday timescales with dispatch and pricing determined through an auction style bidding format. Participants in these markets are balance responsible for their deviations between offered (accepted) supply and actual deliveries. For conventional power plants with controllable fuel resources, these imbalances are usually insignificant. However the variable non-controllable nature of the atmosphere combined with errors in the longer-term wind power forecasts can lead to large financial losses in cases where the direction of the imbalance is unfavourable to the grid. The forecasts for unsubsidized wind power players in day-ahead markets are normally generated by numerical weather prediction (NWP) models, with corrections being made through the intraday markets to reflect updated predictions about real production. The lead time to gate closure for these short term markets is arbitrarily chosen by the market operator. Traditionally power has been traded in 1-hour blocks, but is beginning to be reformulated into blocks of 5, 10, 15, or 30 minutes with gate closures also on the minute-scale as this reduces balancing costs and thus pricing to consumers. The utility of remote sensing based forecast models is therefore also applicable in this field, with the main stakeholders being plant operators, energy traders and energy asset managers.

The fundamental variability of the wind on very-short time scales justifies the need for real-time measurement inputs to minute-scale forecasting models. Physical-computational approaches are limited by their knowledge of boundary conditions and ability to deliver results with the required lead time. Alternatively, time series modelling approaches are able to exploit the strong temporal autocorrelation of the wind to demonstrate skill on the minute scale. However they are limited by their ability to only infer patterns from historical data. Therefore, a remote sensing based approach has been envisioned which can measure spatially distributed winds with a high sampling rate. The added value in this approach is that upwind measurements constitute the wind resource which is advecting towards the site. The local observations can then be used as inputs to a forecasting model.

To gather data for testing and evaluating the potential for remote sensing based forecasts, a series of field experiments were conducted using DTU's fleet of pulsed scanning Doppler lidars. The initial WAFFLE investigation (Section \ref{sec:waffle}), which used a ground-based unit in plan position indicator (PPI) configuration to measure winds with a low elevation angle, saw a breakdown in correlations at distance of the horizontal winds due to the effect of wind shear and the natural vertical decorrelation of winds.
%vertical motion within the probe volume.
This was rectified in the following {\O}sterild Balconies field campaign (Section \ref{sec:balcony_intro}) by mounting the lidar alongside a meteorological mast at the desired height and scanning along a flat horizontal plane. The 1-D wind retrieval correlation results from the Balconies experiment demonstrated a sharp correlation peak with a spatiotemporal relationship extended out to 3 km upwind before significantly broadening. In the final LASCAR field trial (Section \ref{sec:lascar_intro}), a similar setup was used where the lidar was deployed on the roof of a building and performed flat, rapid PPI scans focusing up to 3 km upwind. 2-D correlations of this dataset were inspected, which show the ability to track coherent spatial features of the upstream wind field as they advect downstream in time, over a timescale of up to 5-minutes.

Two classes of remote sensing based forecasts have been identified. The first being classification based approaches which can be used to detect extreme events and warn of incidents expected to occur but at an unknown time. An example of such an event is the weather front passage with 180\degree \ direction change captured during the Balconies experiment at {\O}sterild (Section \ref{sec:balcony_addendum1}). The boundary was first detected 2-hours before arrival to the site, and its propagation was tracked over the same period. Furthermore, the event was not contained in the NWP (WRF) forecasts yet would have a substantial impact on both the energy production and loads of the turbines. This type of forward information is invaluable as statistical models based on historical data have no skill in correctly predicting these types of events.

The second category represents regression based approaches which give a deterministic or probabilistic forecast output minutes-ahead. This has been the main area of focus in this body of work. The first technique, applied on the WAFFLE dataset, is a basic approach which shifts the IVAP reconstructed upstream wind speed signal by the time-of-flight distance to the downstream reference position. This method (at the 1-minute horizon) has achieved a 20\% and 30\% reduction in RMSE over persistence for wind speed and wind power respectively. 

The following study, applied on the {\O}sterild Balconies dataset, uses a wind direction aligned retrieval method to obtain upwind wind speeds which are fit to a linear model using stochastic gradient descent regression (SGDR). The model is incrementally fit following each newly available measurement, and a range of 1 to 60-minute ahead wind speeds are predicted. This method has achieved similar improvements over (10-minute average) persistence on the 1-minute scale (21\%), 10.9\% at the 5-minute scale, 9.2\% at 10-minutes, 7.1\% at 30-minutes, and 6.2\% at 60-minutes. 

The final and most complex technique was applied on the LASCAR dataset. The 2-D radial speeds were input into a convolutional-recurrent neural network (ConvLSTM) which exploits spatiotemporal patterns from sequences of lidar scan images to forecast wind vectors in a multi-output fashion up to 5-minutes ahead. The wind vector approach was used in order to extend the forecasts to include wind direction. This had not been incorporated into the previous studies because its importance for wind farm control had not yet been established. At the 1-minute ahead timestep, performance is similar to the first two lidar based techniques: 18\% and 21\% RMSE improvement over persistence for wind speed and wind direction respectively. However, the ARIMA time series model benchmark demonstrates statistically higher skill for time horizons greater than 30-seconds, leading to questions of the added value and justification of the resource requirements needed to deploy and operate the instrument for routine forecasting purposes (i.e. not extreme events).
%compared to advanced approaches like ARIMA. 

Overall, remote sensing instruments including long-range pulsed Doppler lidars present a promising resource for inputs to generating minute-scale wind power forecasts. They provide spatial and temporal information with a high level of detail about impending conditions and can detect changes in the wind resource before arrival to the site. This can take the shape of either an anomaly classification system for extreme events, or as a routine (continuous) forecasting tool which can be used in advanced control and trading algorithms. The former provides clearer added benefits, as there are no existing methods to attain this information on the minute scale with high accuracy (particularly the timing of such events). The latter has been applied to produce wind speed and wind vector (speed and direction) forecasts, with the vector approach preferable due to its usefulness in wind farm control and in the transformation of wind resources to wind farm power output. Derived products such as the wind direction variance (meandering) and wind speed variance (a measure of turbulence) are also available using this method, which are relevant for use in wind farm models. The worthiness of the continuous approach is highly dependent on a range of factors including; local meteorological and topographical conditions at the site (e.g. site specific weather patterns, on/off-shore deployment), the scale of the wind farm (e.g. spatial, layout, installed capacity), economic constraints (e.g. market mechanisms, balancing penalties, cost of instrumentation and maintenance), and technical characteristics (e.g. model formulation, data quality control, integration with other processes). 

Ultimately, as the next generation of wind power forecasting methods are developed, the improved accuracy and widespread use of minute-scale practices will lead to the more efficient utilization of wind power and thereby a reduction in the levelized cost of electricity.

