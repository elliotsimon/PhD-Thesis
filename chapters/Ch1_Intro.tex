%!TEX root = ../Thesis.tex
\chapter{Introduction}
\label{sec:intro}

%--------------------------------------------------------------------------------
\clearpage
\section{Energy generation from the wind}
\label{sec:intro_engen}


%--------------------------------------------------------------------------------
\subsection{Motivation and global status}
\label{sec:intro_history}

Global energy systems are currently undergoing a revolution. The production of electricity for household and industrial use has, until the recent past, relied almost entirely on non-renewable fuel sources including coal and petroleum derivatives which are burned to run steam turbine generators. A number of compelling motives exist which are precipitating change to the status quo.

The first motivation being the overwhelming scientific evidence of the impacts of large scale releases of air pollutants into the environment. Particulates have been closely linked to lung (\cite{hamra_outdoor_2014}) and heart disorders (\cite{du_air_2016}). Carbon dioxide, a greenhouse gas, is also released through the combustion process and acts to increase the Earth's surface temperature through radiative forcing (\cite{charlson_climate_1992}), as well as acidify water bodies through the formation of carbonic acid and harming marine life (\cite{doney_ocean_2009}). Other pollutants including sulfur oxides and nitrogen oxides contribute to smog and acid rain, and are similarly linked to acute heath effects (\cite{brunekreef_air_2002}).

Realizing the urgency of these environmental concerns, authorities across the world have begun to enact agreements to curb their emissions contributions. These pacts can range from city and regional planning regulations, to national legislation and international treaties. Examples include the United Nations Kyoto Protocol and Paris Climate Agreement, European Commission's Clean Energy for all Europeans Framework (\cite{ec_clean_energy}), China's Renewable Portfolio Standard (\cite{china_rps}) and Denmark's Energy Strategy (\cite{danmark_energi}). In conjunction with improvements in energy efficiency, large impacts can be made by the exchange and supplementation of low-carbon, renewable based generation including wind and solar.

Beyond commitments to avoiding the negatives associated with hydrocarbon based fuels, new opportunities have emerged with the maturation specifically of the wind energy industry. Rapid cost decreases have been demonstrated through experience, competition, and scaling which have brought the levelized-cost-of-energy (LCOE) of onshore wind power within reach and in many areas below that of traditional power plants without the need for subsidies. The IRENA renewable cost database reveals a 2017 average global LCOE for onshore wind at 60 and offshore wind at 140 USD/MWh, with projections for further decreases in the coming years (\cite{IRENA_2018}). The 2019 tender for Saudi Arabia's first installation, the 400 MW Dumat Al Jandal wind farm, resulted in a record power purchase agreement (PPA) of 21.3 USD/MWh (\cite{masdar_edf_2019}).

A final rationale for embracing renewable energy development in many areas is the opportunity for leveraging locally available resources instead of relying on imported fuels from territories which are often rife with geopolitical conflicts. Ensuring security of supply for the local population through greater self-sufficiency can lead to more a balanced hegemony in regional and global politics.

While wind power only supplies about 4\% of global electricity demand at present (14\% in the European Union and 44\% in Denmark), the aforementioned developments have resulted in a large and rapid expansion of planned and installed projects worldwide. Figure \ref{fig:wind_power_cum} presents a timeline of cumulative globally installed wind power, with the 2017 aggregate totalling over 539 GW. The growth rate is projected to remain near 10\%, year on year (\cite{gwec_global_2017}) with many large projects already approved and financed for construction in the coming years.

\begin{figure}[htbp]
    \centering
        \includegraphics[width=1.0\textwidth]{graphics/intro/motivation_market/wind_power_cum.png}
    \caption{Cumulative worldwide installed wind power capacity from 2001-2017. The current value exceeding 539 GW. Source: \cite{gwec_global_2017}}
    \label{fig:wind_power_cum}
\end{figure}

%--------------------------------------------------------------------------------
\clearpage
\subsection{Intermittency of wind resource}
\label{sec:intro_intermittency}

The reliable and efficient exploitation of wind power faces a unique set of challenges. Electrical energy produced by wind turbines is derived from the kinetic energy of the wind. Air flow generates a lift force on the blades causing them to rotate. This mechanical energy ultimately drives the generator, producing electricity which is collected and fed to the power grid. As the wind is not a controllable fuel source, the turbine's output will to a large extent be determined by atmospheric conditions.

Winds originate from differential heating of the Earth's surface and are transported by bulk motion. Within the boundary layer, they are largely influenced by the planet's surface (terrain and vegetation), human made obstacles, weather systems, and turbulent mixing.

Wind variability is defined as the fluctuations in energy content of the wind. These variations occur across a wide range of temporal and spatial scales. Contributions can include diurnal, seasonal, and interannual patterns, and physical processes including: gravity waves, cold fronts, storms, cellular convection, convective rolls, low level jets, and sea breezes (\cite{vincent_forecasting_2017}). Combinations of these synoptic-, meso- and micro-scale influences lead to a high degree of intermittency in the wind, as shown in Figure \ref{fig:wind_speed_power_ts} using 1-second measurements.

Analysis of the power spectral density (PSD) demonstrates the distribution of energies which contribute to the signal variability by frequency. An example is shown in Figure \ref{fig:wind_speed_power_psd} using the same high-frequency meteorological and turbine data. The results are obtained using Welch's method (\cite{welch_use_1967}) with Hamming windowing to reduce noise in the spectra.

\begin{figure}[htbp]
    \centering
        \includegraphics[width=1.0\textwidth]{graphics/intro/variability/wind_speed_power_ts.png}
    \caption{Time series example of wind speed (top) and wind power (bottom) variability from DTU's V52 research turbine}
    \label{fig:wind_speed_power_ts}
\end{figure}

\begin{figure}[htbp]
    \centering
        \includegraphics[width=1.0\textwidth]{graphics/intro/variability/wind_speed_power_psd.pdf}
    \caption{Power spectral density (PSD) of wind speed (left) and wind power (right) from DTU's V52 research turbine}
    \label{fig:wind_speed_power_psd}
\end{figure}

%\cite{larsen_full_scale_2016} 

%The variability of the wind 
%Winds consist of a dominant wind force but also irregular fluctuations (turbulence)
%varying levels of turbulence 
%turbulent kinetic energy (TKE) driven by atmospheric stability
% other interactions such as with thermal convective up/downdrafts and turbulent eddies
%Wind motion exists Global atmospheric circulation originates from differential heating of the Earth's poles and equator which is rotated by the Coriolis force due to the planet's spin. Large scale circulation cells 
%Winds are characterized by their velocity (speed and direction) and 

TODO:
\begin{itemize}
\color{red}
    \item scales, turbulence etc.
    \item power curve
    \item wind ramps
    \item averaging over rotor area and wind farm area (though highly correlated)
    \item the most rapid variations will to some extent be compensated for by the inertia of the wind turbine rotor.
    \item large number of turbines are usually installed within a small geographical area, leading to correlated fluctuations between many turbines
    \item flexible dispatch, higher predicted variability -> more reserve wind power (curtailment)
    \item balancing in high penetration
    \item there are ideas for storage but none implemented at large scale at this time
\end{itemize}

%--------------------------------------------------------------------------------
\clearpage
\subsubsection{Case study of wind turbine power variability}
\label{sec:intro_intermittency_V52}

To further explore wind variability and its impact on power systems, a summary investigation was conducted to characterize changes in electrical power output from a real world wind turbine. The data consists of high-resolution SCADA measurements from DTU's Vestas V52 research turbine at Ris{\o} (\cite{dtu_v52}). This model is one of the most commonly sold turbines worldwide and has a rated capacity of 850 kW. The sourced data spans from March 30 to May 4th, 2016 (35 days) during a calibration period when the turbine's control systems were under normal operation and no aerodynamic modifications were present. 

Measurements of the turbine's active power signal were down-sampled from 35 Hz to 1-second averages, and normalized with respect to rated power (where a value of 100 represents the generator's nameplate capacity). Note that it is possible to have both values below zero (during start up when drawing power from the grid) as well as values above 100 on the short term.

Statistics of absolute changes in the turbine's normalized power output within various time frames ranging from 1-second to 1-hour are presented in the following. Figure \ref{fig:act_pow_change_vioboxplot} presents a combined violin and boxplot across the selected time windows which illustrates statistical properties such as the shape and width of the distributions, quartile positions, median values, and spread of outliers. This is joined with Table \ref{tab:intro_v52_variability_statistics} describing summary statistics.

\begin{figure}[htbp]
    \centering
        \includegraphics[width=1.0\textwidth]{graphics/intro/variability/act_pow_change_vioboxplot.png}
    \caption{Combined violin and boxplot of normalized active power changes by various time windows from DTU's V52 research wind turbine}
    \label{fig:act_pow_change_vioboxplot}
\end{figure}

\begin{table}
    \centering
    \caption{Table of statistics for V52 power output variability over selected time windows up to 1-hour}
    \resizebox{\columnwidth}{!}{%
\begin{tabular}{lrrrrrrrrr}
\toprule
{} &          1 s &          5 s &         30 s &         60 s &        300 s &        600 s &        900 s &       1800 s &       3600 s \\
\midrule
count &  2976599 &  2976595 &  2976570 &  2976540 &  2976300 &  2976000 &  2975700 &  2974800 &  2973000 \\
mean  &        0.000 &        0.000 &        0.000 &        0.000 &        0.002 &        0.003 &        0.004 &        0.008 &        0.009 \\
std   &        1.209 &        4.387 &        8.133 &        9.536 &       11.362 &       11.976 &       12.532 &       13.674 &       15.485 \\
min   &      -26.250 &      -54.907 &      -76.028 &      -84.816 &     -106.745 &     -105.303 &     -106.126 &     -104.044 &     -103.887 \\
25\%   &       -0.225 &       -0.888 &       -1.841 &       -2.103 &       -2.630 &       -3.075 &       -3.371 &       -4.049 &       -4.957 \\
50\%   &        0.000 &        0.000 &        0.000 &        0.000 &        0.000 &        0.000 &        0.000 &        0.000 &        0.000 \\
75\%   &        0.214 &        0.782 &        1.515 &        1.860 &        2.544 &        2.972 &        3.341 &        4.064 &        4.870 \\
max   &       32.326 &       74.414 &      101.242 &      101.663 &      106.567 &      106.013 &      105.350 &       99.266 &      102.539 \\
\bottomrule
\end{tabular}
}
    \label{tab:intro_v52_variability_statistics}
\end{table}

As expected, the variability grows with the length of the window. In all cases, the mean and median power output change is very close to zero and probability densities are symmetrical (normally distributed). On the very shortest timescales (1 and 5-seconds), the variations within the interquartile range (IQR) are small. However, from 30-seconds to 1-minute windows, the spread grows considerably. By the 5-minute case (300 s), the standard deviation approaches that of the longer timescales.

This is further shown in Figure \ref{fig:norm_act_pow_error_dist}, where the distributions are stacked atop each other for comparison (note the logarithmic y-axis scaling). Tail bumps present for certain time windows near the peripheries indicate periods of automatic start up (right tail) when the cut-in wind speed is reached and shut down (left tail) when the cut-out wind speed is exceeded.

\begin{figure}[htbp]
    \centering
        \includegraphics[width=1.0\textwidth]{graphics/intro/variability/norm_act_pow_error_dist.png}
    \caption{Distribution of changes in normalized active power over various time horizons from DTU's V52 research wind turbine}
    \label{fig:norm_act_pow_error_dist}
\end{figure}

This simplified investigation has considered a single wind turbine and not collective wind farm output or otherwise geographically distributed generation which will act to some degree as a smoothing filter. Having said that, the case study has demonstrated that minute-scale variability of wind power is not insignificant and attention should also be focused on this timescale alongside the more commonly focused periods (e.g. 1-hour).

%--------------------------------------------------------------------------------
\clearpage
\subsection{Wind farm control}
\label{sec:intro_control}

Utility scale wind power installations consist of multi-turbine arrangements known as wind farms. The layout is normally characterized by rows of turbines which affect each other through complex aerodynamic interactions. The most notable being wake induced power losses (through a reduction in the extractable energy of the wind by upstream turbines) and fatigue loading (by increased turbulence originating from the air-blade interaction of upstream turbines).

Modern wind turbines are designed with control systems which act to optimize their performance from an individual perspective. However, all together this greedy approach does not always result in the best coordination of the wind farm as a whole. The field of wind farm control aims to collectively optimize the power and loads of the entire wind farm in a collaborative manner by orchestrating control over each turbine's set-points. The overall objectives being to either maximize total active power or provide power control by following a reference signal, and to minimize fatigue loads (\cite{knudsen_survey_2015}). The two main mechanisms for dynamic wind farm control are induction control and wake steering.

Axial induction control reduces the velocity deficit of the downstream wake leading to larger amounts of recoverable energy available to downstream turbines. A wind turbine's power coefficient at its maximum operating point is much less sensitive to changes in pitch angle than its thrust coefficient (\cite{annoni_analysis_2016}). Exploiting this relationship allows for small decreases in the upstream turbine's production to be more than made up for by increased production at the downstream turbines. This is accomplished by slightly pitching the rotor outwards (towards feathered position) and adjusting the generator torque which results in a decrease in the power and thrust coefficients and a reduction in the tip speed ratio. Normally this is performed by reducing the active power of the turbine (downrating), instead of manually interfacing with the turbine's pitch and torque set-points. \cite{gebraad_maximum_2015} establishes a 1.36\% improvement in annual energy production (AEP) using dynamic induction control simulations at the Princess Amalia Wind Farm in the Netherlands.

Wake steering is the intentional misalignment of an upstream turbine's rotor in order to redirect (steer) the wake away from downstream turbines. During normal operation, the turbine will orient its rotor perpendicularly to the wind direction. Applying wake redirection causes the turbine's yaw motor to re-orient itself to an angle relative to the wind direction (Fig. \ref{fig:wake_steering}). This causes the wake to deflect away from downwind turbines, leading to a net increase in energy production for the overall wind farm. Positive yaw misalignment angles are normally used, as this results in a positive tilt angle which also directs the wake downwards. Simulations have indicated the potential for power gains using this approach while also avoiding significant load repercussions. The gains in AEP are on the order of 5\% through the reduction of wake losses (\cite{knudsen_survey_2015}, \cite{gebraad_maximization_2017}). The concept has been successfully demonstrated in a recent field trial (\cite{fleming_field_2017}) which confirms the simulation estimates.

\begin{figure}[htbp]
    \centering
        \includegraphics[width=0.5\textwidth]{graphics/intro/wake_steering.png}
    \caption{Comparison of the horizontal wake during normal operation (top panel) and under wake steering control (bottom panel, 30\degree \ yaw misalignment). The actual wake center-line is indicated in black, with the linear downstream reference in pink. Simulation using NREL's SOWFA toolbox in OpenFOAM. Source: (\cite{fleming_wakesteering_2014})}
    \label{fig:wake_steering}
\end{figure}

Wind farm control algorithms require estimates of the incoming wind speed and direction. These inputs are combined with a delay factor to account for the wind field advection between turbine rows. The forecasts are applicable on the minute scale for preemptive optimization by the wind farm controller and for the individual turbines to adjust their configurations accordingly. Induction control is sensitive to accurate wind speed inputs, while wake steering is sensitive to accurate wind direction inputs. Improved forecast performance on this timescale therefore also enhances the benefits of dynamic wind farm control.

The field of individual turbine control (i.e. feed-forward and model predictive control) is largely irrelevant in the context of this project, as the timescales are in the order of seconds ahead where direct advection (frozen turbulence) models perform sufficiently well. There may be a case for individual turbine yaw control on the minute scale, however this is seen as a low priority application with limited relative benefits over existing solutions.

%--------------------------------------------------------------------------------
\clearpage
\subsection{Forecasting for wind energy}
\label{sec:intro_forecasting}

The evolution of wind power together with advancements in computing and an improved understanding of atmospheric dynamics has established forecasting as a central element of operational research. An entire sub-industry is dedicated to developing and providing forecasting services to end users which include: wind farm owners and operators, power traders, asset managers, and transmission system operators (TSOs).

Wind prediction is conducted on a wide range of time scales, with lead times spanning from a few milliseconds up to one week or more. Prediction intervals can be described by a few broad categories, each with their own approaches and applications. Table \ref{tab:forecast_overview_table} outlines the common forecast horizons relevant to wind energy and typical methods applied within them. Broad overviews of the field are also presented in \cite{costa_review_2008}, \cite{giebel_state---art_2011} and \cite{soman_review_2010}.

\begin{table}
    \centering
    \caption{Overview of forecast intervals of interest for wind energy purposes}
    %\resizebox{\textwidth}{!}{%
\begin{tabular}{|l|l|l|l|l}
\large
\cline{1-4}
Designation & \textbf{Typical horizon} & \textbf{Example methods} & \textbf{Example applications} &  \\ \cline{1-4}
\textbf{Immediate} & \begin{tabular}[c]{@{}l@{}}Milliseconds\\ to seconds\end{tabular} & \begin{tabular}[c]{@{}l@{}}--- Persistence\\ --- Wind field measurements using nacelle lidars {[}1{]}\\     and/or upwind turbine SCADA {[}2{]}\end{tabular} & \begin{tabular}[c]{@{}l@{}}--- Wind turbine control {[}1{]}\\ --- Grid regulation {[}3{]}\\     (e.g. frequency, voltage support)\end{tabular} &  \\ \cline{1-4}
\textbf{\begin{tabular}[c]{@{}l@{}}Very short-term \\ (minute scale)\end{tabular}} & \begin{tabular}[c]{@{}l@{}}1-minute\\ to 1-hour\end{tabular} & \begin{tabular}[c]{@{}l@{}}--- Persistence {[}4{]}\\ --- Statistical time series models {[}5{]}\\ --- Markov (regime switching) models {[}6{]}\\ --- Machine learning and artificial neural networks (ANN) {[}7,8{]}\end{tabular} & \begin{tabular}[c]{@{}l@{}}--- Wind farm control\\ --- Ancillary services \\     (e.g. reserve power) {[}2,9{]}\\ --- Intrahour energy market trading {[}10{]}\\ --- Storage management\\     (e.g. battery storage control)\end{tabular} &  \\ \cline{1-4}
\textbf{Short-term} & 1 to 72 hours & \begin{tabular}[c]{@{}l@{}}--- Statistical time series models {[}11,12{]}\\ --- Numerical weather prediction (e.g. WRF) {[}13{]}\\ --- Analogue ensemble prediction {[}13,14{]}\\ --- Kalman filter {[}11,15{]}\end{tabular} & \begin{tabular}[c]{@{}l@{}}--- Intraday and day-ahead energy market trading {[}10{]}\\ --- Ancillary services\\ --- Storage management\\     (e.g. battery, hydrogen and pumped storage control) {[}16{]}\\ --- Economic dispatch and generator planning\\ --- Operator portfolio management\end{tabular} &  \\ \cline{1-4}
\textbf{Long-term} & \begin{tabular}[c]{@{}l@{}}72 hours \\ to 10 days or more\end{tabular} & \begin{tabular}[c]{@{}l@{}}--- Same as short term\\ --- Climatology\end{tabular} & \begin{tabular}[c]{@{}l@{}}--- Reserve requirement decisions\\ --- Unit commitment decisions\\ --- Maintenance scheduling\end{tabular} &  \\ \cline{1-4}
\end{tabular}
}
    \includegraphics[width=1.0\textwidth]{tables/forecast_overview_table.pdf}
    \resizebox{\textwidth}{!}{%
\begin{tabular}{|c|}
\hline
\textbf{References} \\ \hline
\begin{tabular}[c]{@{}c@{}}
{[}1{]} \cite{schlipf_nonlinear_2012}, 
{[}2{]} \cite{gocmen_possible_2016}, 
{[}3{]} \cite{hansen_provision_2016}, 
{[}4{]} \cite{hodge_wind_2011}, 
{[}5{]} \cite{pinson_very_2012}, 
{[}6{]} \cite{trombe_general_2012}, \\ 
{[}7{]} \cite{potter_very_2006}, 
{[}8{]} \cite{niu_multi-step-ahead_2018}, 
{[}9{]} \cite{mackenzie_short-term_2017}, 
{[}10{]} \cite{bathurst_trading_2002}, 
{[}11{]} \cite{liu_comparison_2012}, \\ 
{[}12{]} \cite{torres_forecast_2005}, 
{[}13{]} \cite{mahoney_wind_2012}, 
{[}14{]} \cite{delle_monache_probabilistic_2013}, 
{[}15{]} \cite{bossanyi_short-term_1985}, 
{[}16{]} \cite{castronuovo_integrated_2013}
\end{tabular} \\ \hline
\end{tabular}
}
    \label{tab:forecast_overview_table}
\end{table}

Forecasting techniques can be categorized into two fundamental classes- process driven physical models and data driven statistical approaches. Here general introductions are given for both. An extensive review is provided in Section \ref{sec:IEA_paper}.

Physical approaches such as numerical weather prediction (NWP) are based on well established physical and mathematical laws. Parameterizations of the atmosphere, where coarse input data (global or synoptic scale) is combined with mathematical modelling of atmospheric properties such as air, soil and sea temperature, pressure, land cover and surface obstacles to provide a local site forecast at varying temporal and spatial resolutions. Optionally, through data assimilation the model states can be iteratively adjusted using real-time observations to adapt the simulation and correct for biases. These systems generally run on large supercomputers and require significant time and computational power to generate their forecasts. Further, they have not sufficiently demonstrated their ability to predict local microscale events that are of greatest relevance for real-time wind farm control. Therefore they are not considered appropriate in the context of minute-scale wind forecasting as they are ill-suited to be used operationally these time scales with today's technology.

Statistical approaches use empirical methods to model relationships between historical observations. The models are then used with real-time inputs to extrapolate future outcomes. Meteorological data is normally given as a time-series, where samples are highly correlated, non-independent and naturally ordered in time. The high temporal autocorrelation of wind measurements lend themselves well to the simplest statistical approach called persistence. Persistence simply forecasts the future value of the series to be the same as the most recent observation, or a moving average of it. This method is widely used on horizons from the immediate to short-range, and is a common benchmark for evaluating more complex techniques. Autoregressive (AR) models are also appropriate, and are widely employed, often in conjunction with moving average (MA) models. In cases where the target signal is non-stationary, a number of transformation steps can be taken to impart stationarity (e.g. differencing, trend removal, seasonal and cyclical adjustments). Formulations which have demonstrated particularly adept skill are autoregressive integrated moving average (ARIMA) models. Further details on ARIMA modelling are given in Section \ref{sec:lascar_paper}. The flexibility of statistical approaches empower their suitability for minute-scale forecasting applications. This thesis work therefore exists within this research topic.

\begin{comment}
Evaluating forecasts?
\end{comment}

%--------------------------------------------------------------------------------
\clearpage
\subsection{Power system and electricity markets}
\label{sec:intro_power_markets}

The vast majority of wind power installations are grid connected and offer the sale of their production through either long term power purchase agreements (PPA) or via wholesale electricity spot markets.

The modern electrical grid is a network of providers and consumers of electrical energy, interconnected via transmission and distribution lines used to transport supply from generation facilities to homes and businesses. Electrical appliances are designed to operate within a narrow range of the design frequency of the local AC power grid. Deviations can result in malfunctions or damage, or even in cascading failures as devices such as pumps, fans and motors drift from their intended output. For this reason, the supply and demand of electricity must be kept in constant balance. Variability occurring on both sides of the equation drives a need for accurate forecasts, which are used in system planning and generator dispatch. Deviations from the equilibrium between generation and consumption requires real-time adjustments, directed by the balancing authority (BA) through the activation of a suite of ancillary services.

Energy markets attempt to provide the optimum allocation of resources through the minimization of pricing while fulfilling balance constraints. This is enacted through an auction style clearing process, where utilities place bids to cover the forecasted demand of their customers, and generators place bids to offer their production into the market. The market price is then set at the intersection of the supply and demand offers, which is calculated for each region (i.e. zone or node). The market is subdivided geographically to account for constraints in transmission capacity, with import/export links considered. Once accepted, participants are balance responsible (BRP) for delivering their scheduled quota within the given trading block. If any deviations occur, they may incur financial penalties depending on the effect of their imbalance (i.e. helping or hurting the system). It is common for market operators (e.g. in NordPool's two-price system) to discourage gaming tactics by preventing revenues from exceeding initial market payouts through exploiting up- and down-regulation price movements.

Market design is highly dependent on location, but exchanges often operate day-ahead and intraday markets with trading segments in one hour blocks, and with long lead times between bid placement and delivery. These lengthy intervals create issues for variable renewables such as wind. Market operators in regions with high penetrations of renewables are moving to shorten these horizons, in certain cases to continuous intraday markets with 5, 15, and 30 minute contracts (\cite{epex_spot_se_continuous_2019}, \cite{aemc_five_2017}) where minute-scale wind forecasts become relevant and necessary for traders.

\begin{comment}
Errors in power forecast do not necessarily reflect impact. Overestimation and underproduction != underprediction and overproduction due to structure of balancing market.
The need for spinning reserves. Frequency control/response mode. Droop speed control. Spinning reserve is extremely expensive for utilities. 
https://en.wikipedia.org/wiki/Dynamic_demand_(electric_power)#The_need_for_spinning_reserve
Cost of imbalance not only money but also CO2
\end{comment}

%--------------------------------------------------------------------------------
%\clearpage
%\subsubsection{Case study of wind power operation in a 5-minute energy market}
%\label{sec:intro_aus_power_study}

%Placeholder for Australian energy market case study


%--------------------------------------------------------------------------------
%--------------------------------------------------------------------------------
\clearpage
\section{Remote sensing}
\label{intro_remote_sensing}

%--------------------------------------------------------------------------------
\subsection{Principles}
\label{sec:intro_rs_principles}

TODO:
\begin{itemize}
\color{red}
    \item Intro
    \item Measurement theory
    \item Introduce lidar and radar
\end{itemize}


%--------------------------------------------------------------------------------
\clearpage
\subsection{Lidar}
\label{sec:intro_lidar}

TODO:
\begin{itemize}
\color{red}
    \item Lidar specific 
    \item Pulsed Doppler lidar measurement chain
    \item Practicalities
    \item Filtering techniques
    \item Interpreting measurements
\end{itemize}


%--------------------------------------------------------------------------------
\clearpage
\subsection{Measurement techniques}
\label{sec:intro_meas_tech}

TODO:
\begin{itemize}
\color{red}
    \item Beam positioning
    \item Switching vs. mechanical
    \item Staring vs. 2D vs. 3D
    \item Fixed vs dynamic trajectories
\end{itemize}

%--------------------------------------------------------------------------------
%\clearpage
%\subsection{Processing, filtering, and interpreting data}
%\label{sec:intro_rs_data}

%TODO:
%\begin{itemize}
%\color{red}
%    \item Data formats
%    \item Filtering techniques
%    \item Interpreting measurements
%\end{itemize}

\clearpage