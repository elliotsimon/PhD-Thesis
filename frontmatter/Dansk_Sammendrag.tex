%!TEX root = ../Thesis.tex
\chapter{Dansk sammendrag}
\label{sec:dansk_sammendrag}


En v{\ae}sentlig hindring for udbredelsen af vindkraft i stor skala er at der optr{\ae}der store variationer i energiproduktion gennem tid grundet vekslende vejrsystemer og turbulens i atmosf{\ae}ren. Fejl i vindprognoser udg{\o}r en v{\ae}sentlig kilde til usikkerhed ved planl{\ae}gning og drift af el systemet, is{\ae}r n{\aa}r de vedvarende energikilder udg{\o}r en stor del af forsyningen. For at l{\o}se dette problem er der i et stigende antal forsyningsmarkeder blevet {\ae}ndret p{\aa} tidsintervallet mellem el handelen og el leverancen. Tidsintervallet er blevet {\ae}ndret fra timer til minutter.

Vindprognoserne p{\aa} minutbasis anvendes i tre sammenh{\ae}nge: Handel inden for en time p{\aa} el markedet; styring af balancen i el systemet; og styring af en hel m{\o}llepark.

Vind- og elkraft-prognoser p{\aa} disse meget korte tidsskalaer opn{\aa}s typisk ved at anvende m{\o}nstre fra tidligere data (tidsserie-modellering) eller ved at antage u{\ae}ndrede forhold fra de seneste tilg{\ae}ngelige m{\aa}linger (persistens). Begge metoder vil under vekslende vindforhold give un{\o}jagtige prognoser, og netop p{\aa} de tidspunkter kan prognoser v{\ae}re s{\ae}rligt betydningsfulde for planl{\ae}gning og drift af el systemet.

Fjernm{\aa}lings-instrumentet Doppler vind lidar kan m{\aa}le vinden flere kilometer v{\ae}k og give observationer med h{\o}j rumlig og tidsm{\ae}ssig opl{\o}sning. N{\aa}r lidaren m{\aa}ler op mod vinden i vindretningen vil de vindm{\o}nstre og strukturer som bev{\ae}ges (advekteres) hen mod instrumentet blive indsamlet i realtid. M{\aa}lingerne kan sammen med atmosf{\ae}riske modeller bruges til at generere stedspecifik vind- og elkraft-prognoser. Det kan udf{\o}res enten ved klassificering af uregelm{\ae}ssige h{\ae}ndelser eller ved en regressionsmetode, der frembringer deterministiske eller statistiske forudsigelser.

For at udforske disse forskellige muligheder er der blevet gennemf{\o}rt en r{\ae}kke eksperimentelle kampagner i l{\o}bet af ph.d.-projektet. Disse er udf{\o}rt i en sekvens, som har givet et referencedatas{\ae}t af h{\o}j kvalitet til anvendelse ved modelformulering og til testning. Mere specifikt er scanning vind lidar observationer blevet anvendt til at unders{\o}ge korrelationer i tid og rum som en funktion af afstanden opstr{\o}ms. Resultaterne viser meget h{\o}j korrelation is{\ae}r inden for de f{\o}rste 3 km opstr{\o}ms for instrumentet. Det svarer til at der findes et optimalt prognosevindue p{\aa} op til 5 minutter. Der er ogs{\aa} mulighed for at observere og forudsige vinden l{\ae}ngere v{\ae}k hvilket giver en l{\ae}ngere tidshorisont for prognosen.

De tre prognosemetoder, som er blevet implementeret, inkluderer en relativ simpel metode hvor vinden antages at bev{\ae}ge sig med samme hastighed som den af lidaren m{\aa}lte horisontale vindhastighed; en line{\ae}r udbredelsesmodel, der bruger vindretningen og vindhastigheder som input; og en avanceret metode med machine learning computer vision, der anvender 2-D lidar-scanninger i et Long Short-Term Memory (LSTM) neuralt netv{\ae}rk. Alle tre lidar-baserede metoder opfylder kravene til at give en prognose, som kan anvendes i realtid. De er alle sammenlignet med en f{\ae}lles benchmark.

Specielle vindforhold observeret under eksperimenterne s{\aa}som pludselige store {\ae}ndringer i vindhastighed (ramps) og passage af vejrfronter er unders{\o}gt n{\o}je for at afklare vind lidarens evne til at opdage og spore disse specielle vindforhold og deres ankomst til m{\aa}lested. 

Den overordnede konklusion af ph.d.-projektet er at vind lidar m{\aa}linger er nyttige i forhold til at lave vind- og elkraft prognoser med. Det er is{\ae}r stor-skala vejrh{\ae}ndelser, som kan m{\aa}les pr{\ae}cist i tid og sted, og hvor de numeriske modeller, der anvendes til vejrudsigter, ikke forudsiger vinden pr{\ae}cist. Forskningen har indeholdt flertrins tidsserier prognoser og disse er med gode resultater sammenlignet statistisk med persistens-metoden. Dog er fordelene af flertrins tidsserie prognoser tvetydige. Tidsserie modelleringsteknikker som ARIMA giver generelt gode resultater. Men kompleksiteten og omkostningerne ved at installere og vedligeholde supplerende instrumentering er muligvis ikke en fordel {\o}konomisk set. I de tilf{\ae}lde hvor der kan v{\ae}re en {\o}konomisk gevinst ved at g{\o}re brugerne i stand til at udnytte marginale forbedringer i prognosens n{\o}jagtighed, s{\aa} kan det overvejes at realisere korttids-prognoser med lidar-baseret m{\aa}ling efter de tekniske retningslinjer, som er beskrevet i denne afhandling.